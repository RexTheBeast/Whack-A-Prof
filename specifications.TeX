\documentclass[11pt]{scrreprt}
% ------------------------------------------------------
% Basic typography and internationalisation
\usepackage[T1]{fontenc}        % modern font encoding
\usepackage[utf8]{inputenc}     % UTF-8 source
\usepackage[english]{babel}     
\usepackage{lmodern}            % improved Latin Modern fonts
\usepackage{microtype}          % better kerning & justification
\usepackage{xspace}            % smart spacing after commands
% ------------------------------------------------------
% Layout
\usepackage{geometry}
\geometry{margin=1in}
% Listings (for any future code extracts)
\usepackage{listings}
\lstset{
  basicstyle   = \ttfamily\small,
  breaklines   = true,
  frame        = single,
  showstringspaces = false
}
% Hyperlinks
\usepackage{hyperref}
\hypersetup{
  pdfusetitle,
  colorlinks  = false,
  pdfborder   = {0 0 0}
}
% ------------------------------------------------------
% Convenience macros
\newcommand*{\product}{\textit{Whack-a-Prof}\xspace}
\def\version{1.1}
\date{\today}
% ------------------------------------------------------
\begin{document}

% =============== Title page ===============================================
\begin{titlepage}
  \centering
  {\Huge\bfseries Software Requirements\\[4pt] Specification\par}
  \vspace{1.5cm}
  {\LARGE \product\par}
  \vfill
  {\Large Version \version\par}
  \vspace{1cm}
  {\large Prepared by Team 2 – Specifications Group\par}
  {\large CISC 3140 Project • Brooklyn College\par}
  {\large \today}
  \vfill
  \rule{\linewidth}{0.5mm}
\end{titlepage}

\tableofcontents
\clearpage

% ====================== 1  INTRODUCTION ====================================
\chapter{Introduction}

\section{Purpose}
This document specifies the requirements for the browser-based game \product, covering functionality, user interfaces, constraints, and external interactions.

\section{Document Conventions}
The structure follows IEEE Std 830-1998 (SRS).

\section{Intended Audience and Reading Suggestions}
\begin{itemize}
  \item \textbf{Development Team}: Chapters 2–5
  \item \textbf{QA Testers}: Chapters 3–5
  \item \textbf{Evaluators}: All chapters
\end{itemize}

\section{Project Scope}
\product is an arcade-style browser game inspired by \emph{Whack-a-Mole}. Players earn points by hitting professors as they pop up from the holes. The game was developed for CISC 3140 at Brooklyn College.

\section{References}
\begin{itemize}
  \item IEEE SRS Standard 830-1998
  \item K. Wiegers, “Software Requirements,” \url{http://karlwiegers.com}
\end{itemize}

% ====================== 2  OVERALL DESCRIPTION =============================
\chapter{Overall Description}

\section{Product Perspective}
\product is a standalone, client-side web application built with HTML5, JavaScript, and CSS.

\section{Product Functions}
\begin{itemize}
  \item Start, pause, and end gameplay
  \item Score points by clicking characters, including professors and trustees
  \item Randomised character appearance
  \item Local-storage leaderboard (highest score)
  \item Customizable mallet cursor (selectable at game start)
  \item Special “trustee” character with unique explosion animation and bonus points when hit
\end{itemize}

\section{User Classes and Characteristics}
\begin{itemize}
  \item \textbf{Primary}: Project evaluators / professors
  \item \textbf{Secondary}: QA testers
  \item \textbf{Tertiary}: Development team
  \item \textbf{End-users}: General players
\end{itemize}

\section{Operating Environment}
\begin{itemize}
    \item \emph{Hardware}: PC, laptop, or mobile device capable of running a modern web browser, equipped with mouse, trackpad, or touchscreen input, and audio output capability.
    \item \emph{Software}: A modern web browser supporting HTML5, CSS3, and JavaScript (See Section 2.7 for specific target browsers and versions).
    \item \emph{Display Requirements}:
        \begin{itemize}
            \item \textbf{Responsive Layout}: The game utilizes a responsive design with dynamic scaling. The user interface elements, particularly the game board, dynamically adapt to the available browser viewport size.
            \item \textbf{Minimum Usable Viewport}: While the layout adapts fluidly, a minimum viewport size of 375 $\times$ 667 pixels (typical older portrait smartphone) is recommended to ensure comfortable interaction and readability. Functionality on significantly smaller viewports is not guaranteed.
            \item \textbf{Physical Size Limitations}: The game requires sufficient physical screen size for accurate targeting. Devices with physical displays smaller than 4 inches (diagonal) are not supported, regardless of resolution or scaling techniques. Even with responsive design, extremely small displays (such as 2 $\times$ 2 inch screens) provide inadequate target sizes for the precision required in gameplay.
            \item \textbf{Pixel Density}: The application is designed to render correctly on both standard-resolution and high-DPI displays (such as Apple Retina displays).
        \end{itemize}
    \item \emph{Audio Requirements}:
        \begin{itemize}
            \item \textbf{Output Device}: The system must have functional audio output capability to experience the full game.
            \item \textbf{Browser Audio Support}: The browser must support the HTML5 Audio API.
            \item \textbf{Note}: The game is playable without audio, but this is not the intended experience.
        \end{itemize}
\end{itemize}

\section{Design and Implementation Constraints}
\begin{itemize}
  \item Implemented entirely in JavaScript (approved libraries permitted)
  \item Source repository hosted on RiouxSVN, accessible at \url{https://svn.riouxsvn.com/whackgame}
\end{itemize}

\section{User Documentation}
\begin{itemize}
  \item In-game interactive tutorial
  \item Contextual help prompts / tooltips
\end{itemize}

\section{Assumptions and Dependencies}
\begin{itemize}
  \item JavaScript and local-storage enabled in browser
  \item Target browsers:
    \begin{itemize}
      \item Chrome 135+
      \item Firefox 137+
      \item Safari 17.x+
      \item Edge 135+
    \end{itemize}
  \item External libraries may be adopted later (TBD)
\end{itemize}

% ====================== 3  EXTERNAL INTERFACES =============================
\chapter{External Interface Requirements}

\section{User Interfaces}
\begin{itemize}
  \item \textbf{Home Screen:} 
    \begin{itemize}
        \item The text ``Whack A Prof!'' will be at the top center of the screen.
        \item Start button: Positioned dead center of the screen.
        \item Tutorial button: Slightly below the Start button.
        \item Leaderboard button: Slightly below the Tutorial button.
        \item Images of the professor on the left and right sides of the screen.
    \end{itemize}
  \item \textbf{Tutorial:}
    \begin{itemize}
        \item Clicking the Tutorial button will open a video or images (based on the graphic designer's choice) that explain how to play the game.
    \end{itemize}
  \item \textbf{Leaderboard:}
    \begin{itemize}
        \item Clicking the Leaderboard button will display a scrollable list of the top 100 scores.
        \item The user's highest score will be shown at the very top of the leaderboard, highlighted and ranked.
    \end{itemize}
  \item \textbf{Start/Weapon Selection:}
    \begin{itemize}
        \item When the user clicks the Start button, a variety of mallet weapons (graphic designer's choice) are displayed in the center of the screen.
    \end{itemize}
  \item \textbf{Game Field:}
    \begin{itemize}
        \item The game field with 16 holes (4x4) will be displayed, spreading from left to right of the screen when a mallet is selected.
        \item The professor will randomly pop up from the holes.
    \end{itemize}
  \item \textbf{Dynamic Timer and Score Display:}
    \begin{itemize}
        \item Dynamic timer: Positioned at the top-left of the screen.
        \item Score display: Slightly to the right of the timer.
    \end{itemize}
  \item \textbf{Pause/Resume and Exit Buttons:}
    \begin{itemize}
        \item Pause/Resume icon button: Positioned at the top-right of the screen.
        \item Exit/End game icon button: Slightly to the right of the Pause/Resume icon button.
    \end{itemize}
  \item \textbf{Mute Button:}
    \begin{itemize}
        \item Toggle Mute button: Positioned slightly to the left of the Pause/Resume icon button.
    \end{itemize}
  \item \textbf{Game Over Screen:}
    \begin{itemize}
        \item A square box will appear in the center of the screen, displaying both the current score and your best score.
        \item Below the score, there will be two buttons: 
        \begin{itemize}
            \item Home button: Returns the user to the home screen.
            \item Play again button: Allows the user to play again with the current score reset to zero.
        \end{itemize}
    \end{itemize}
\end{itemize}

Sketches and mock-ups will be supplied separately.

\section{Hardware Interfaces}
\begin{itemize}
  \item Mouse / track-pad
  \item Touchscreen
\end{itemize}

\section{Software Interfaces}
\begin{itemize}
  \item HTML5, CSS3, JavaScript libraries
  \item Browser Local Storage API
\end{itemize}

\section{Communication Interfaces}
None (client-side only).

% ====================== 4  SYSTEM FEATURES =================================
\chapter{System Features}

\section{Gameplay and Scoring Mechanics}

\subsection{Description}
A fast-paced game in which professors randomly pop up from any of the 16 holes. Players click them to earn points; an on-screen score updates immediately. Top scores persist locally.

\subsection{Stimulus/Response Sequences}
\begin{enumerate}
  \item Door opens; professor character appears.
  \item Player clicks / taps character.
  \item Game increments score.
  \item Successful hit: +10 points.
  \item Miss or inactivity: –5 points.
  \item Trustee character (appearing approximately once per game) triggers a brief explosion animation ($\approx$ 1 s) and awards +20 points.
\end{enumerate}

\subsection{Functional Requirements}
\begin{itemize}
  \item \textbf{REQ-1.1}: Characters appear at uniformly random intervals of 0.5–1.5 s.
  \item \textbf{REQ-1.2}: Trustee character must appear randomly with approximately 5-10% probability. Trustee must trigger a special explosion animation visibly overlaying the screen for $\approx$ 1 s and play an accompanying scream sound effect.
  \item \textbf{REQ-1.3}: Characters vanish after 2 s if not clicked.
  \item \textbf{REQ-1.4}: Player's cursor must be visually replaced by a selected mallet graphic which appears to "whack" when clicked.
  \item \textbf{REQ-2.1}: Score updates in real-time and after each interaction.
  \item \textbf{REQ-2.2}: Top scores are stored via Local Storage.
  \item \textbf{REQ-2.3}: Sound effect plays on character clicks, misses, and trustee hits. Specific sound to be determined during implementation.
\end{itemize}

\section{Weapon Selection Interface}

\subsection{Description}
Before gameplay begins, players are presented with a weapon selection interface where they can choose their preferred mallet style. This interface enhances personalization and engagement, allowing players to select the whacking implement that best suits their style.

\subsection{Stimulus/Response Sequences}
\begin{enumerate}
  \item Player clicks "Play" on the main menu.
  \item Weapon selection screen appears with visual representation of available mallets.
  \item Player selects desired mallet.
  \item Game begins with the player's cursor replaced by their selected mallet.
\end{enumerate}

\subsection{Functional Requirements}
\begin{itemize}
  \item \textbf{REQ-WS-1}: The game must offer at least two visually distinct mallet options.
  \item \textbf{REQ-WS-2}: Each mallet must have both a selection image and a corresponding cursor representation.
  \item \textbf{REQ-WS-3}: The selected mallet must visually transform (scale down briefly) when clicked to simulate impact.
  \item \textbf{REQ-WS-4}: Player's mallet selection must persist for the entire gameplay session.
  \item \textbf{REQ-WS-5}: Weapon selection interface must include preview images that clearly represent how each mallet will appear during gameplay.
\end{itemize}

\section{Audio and Sound Effects}

\subsection{Description}
The game implements a comprehensive sound system to provide audio feedback for game events and enhance the user experience. All sounds follow a consistent style that matches the game's lighthearted theme.

\subsection{Stimulus/Response Sequences}
\begin{enumerate}
  \item Game start: Plays introductory sound.
  \item Professor hit: Plays "hit" sound.
  \item Miss: Plays "miss" sound.
  \item Trustee hit: Plays unique "explosion" sound with scream effect.
  \item Game over: Plays concluding sound.
  \item New high score: Plays celebratory sound.
\end{enumerate}

\subsection{Functional Requirements}
\begin{itemize}
  \item \textbf{REQ-3.1}: Game must provide audio feedback for all major user interactions and game events.
  \item \textbf{REQ-3.2}: Distinct sounds must play for:
    \begin{itemize}
      \item Game start
      \item Successful professor hits
      \item Missed attempts
      \item Trustee character hits (unique explosion sound)
      \item Game over
      \item Achievement of new high score
    \end{itemize}
  \item \textbf{REQ-3.3}: Sound effects must synchronize with their corresponding visual events with latency $\leq$ 50 ms.
  \item \textbf{REQ-3.4}: Game must include a mute/unmute toggle button that persists user preference across sessions via Local Storage.
  \item \textbf{REQ-3.5}: Volume level must be consistent across all sound effects to prevent unexpected loud sounds.
  \item \textbf{REQ-3.6}: Sound format must be MP3 with WAV fallback for maximum browser compatibility.
  \item \textbf{REQ-3.7}: Individual sound effect files must not exceed 100 KB to ensure quick loading times.
  \item \textbf{REQ-3.8}: Game must remain fully playable with audio disabled for accessibility.
  \item \textbf{REQ-3.9}: In addition to mute/unmute, the game should provide volume adjustment with settings persisted in Local Storage.
  \item \textbf{REQ-3.10}: Audio system must efficiently pre-load and cache sound effects to prevent performance degradation.
  \item \textbf{REQ-3.11}: Game must gracefully handle scenarios where audio playback is not supported or permission is denied.
\end{itemize}

% ====================== 5  NON-FUNCTIONAL REQS =============================
\chapter{Non-functional Requirements}

\section{Performance}
\begin{itemize}
  \item Initial page load $\leq$ 5 s (on broadband).
  \item Animation renders at 60 fps on supported hardware.
  \item Audio playback must begin within 50 ms of triggering events.
\end{itemize}

\section{Security}
No sensitive data processed. All data remain local to the browser.

\section{Software Quality Attributes}
\begin{itemize}
  \item Readable, maintainable codebase
  \item Robust gameplay with graceful error handling
\end{itemize}

\section{Error Handling}
\begin{itemize}
  \item Detect and report Local Storage quota issues.
  \item Provide clear feedback for unsupported browsers.
\end{itemize}

% ====================== APPENDICES =========================================
\appendix

\chapter{Glossary}
\begin{description}
  \item[Professor] Standard clickable target.
  \item[Trustee] Special character with higher point value (20 points) that appears less frequently and triggers a unique explosion animation with scream effect.
  \item[Mallet] The player's cursor, visually represented as a whacking implement. Players can choose between different mallet styles before starting the game.
  \item[FPS] Frames per second.
  \item[Local Storage] Browser-side key-value store.
\end{description}

%\chapter{Analysis Models}
% Placeholder for future UML diagrams or other models.

\chapter{To Be Determined}
\begin{itemize}
  \item Final UI mock-ups and design specifics
  \item Final JavaScript library selection
  \item Precise animation specification for trustee effect
\end{itemize}

\end{document}
